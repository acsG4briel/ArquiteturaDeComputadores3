\documentclass[conference]{IEEEtran}
\IEEEoverridecommandlockouts
% The preceding line is only needed to identify funding in the first footnote. If that is unneeded, please comment it out.
\usepackage{cite}
\usepackage{amsmath,amssymb,amsfonts}
\usepackage{algorithmic}
\usepackage{graphicx}
\usepackage{textcomp}
\usepackage{xcolor}
\def\BibTeX{{\rm B\kern-.05em{\sc i\kern-.025em b}\kern-.08em
    T\kern-.1667em\lower.7ex\hbox{E}\kern-.125emX}}
\begin{document}

\title{Príncipios de Localidade na Cache: Impacto de Técnicas e Arquiteturas na Performance.\\

}

\author{\IEEEauthorblockN{Gabriel Araujo Campos Silva}
\IEEEauthorblockA{\textit{Pontifícia Universidade Católica de Minas Gerais}}
\and
\IEEEauthorblockN{Gabriel Rangel}
\IEEEauthorblockA{\textit{Pontifícia Universidade Católica de Minas Gerais}}}

\maketitle

\section{Introdução}
A memória cache é um componente de Hardware de computadores que armazena dados/instruções e tem rápido tempo de acesso, apesar de seu espaço ser reduzido. Para extrair o máximo de eficiência dessa memória, existem diversas técnicas capazes de otimizar sua performance.

A performance de uma memória Cache é medida através do cache miss, que é uma métrica para mensurar quantas vezes a informação que o processador necessita estava armazenada. É importante ressaltar que cada vez que um dado não é encontrado na Cache, ocorre um acesso às demais memórias, que consequentemente gastam mais tempos para fazer essa busca.

De acordo com Patterson e Hennessy \cite{patterson2017}, a localidade temporal é o princípio em que se um local de dados é referencado, então, ele tenderá aa ser referenciado novamente em breve. Enquanto a localidade espacial é o princípio em que, se um local de dados é referenciado, então, os próximos dados com endereços próximos tenderão a ser referenciados em breve.

Para explorar as os conceitos de localidade, existem diferentes arquiteturas que possibilitam mapear a memória de forma a diminuir ou aumentar a quantidade de substituições que serão feitas ao longo de uma execução. Além disso, os princípios também podem ser analisados de acordo com as políticas de substituição de um dado na memória. Por fim, o tamanho dos blocos e das palavras também interfere na performance e pode perturbar o melhor uso das localidades.

O trabalho apresenta a comparação entre diversas técnicas de substituição, como LRU (Least Recently Used), LFU (Least Frequently Used) e FIFO (First In First Out), além de diferentes arquiteturas de memória, como Mapeamento Direto, Associativa por conjuntos e Completamente Associativa, e como cada alteração impacta o cache miss através de uma análise sobre o funcionamento de cada, comparação entre todos e simulação do funcionamento da memória através do simulador Amnesia.

\section{Trabalhos Correlatos}


\section{Metodologia}
The preferred spelling of the word ``acknowledgment'' in America is without 
an ``e'' after the ``g''. Avoid the stilted expression ``one of us (R. B. 
G.) thanks $\ldots$''. Instead, try ``R. B. G. thanks$\ldots$''. Put sponsor 
acknowledgments in the unnumbered footnote on the first page.

\section{Arquiteturas Propostas}
The preferred spelling of the word ``acknowledgment'' in America is without 
an ``e'' after the ``g''. Avoid the stilted expression ``one of us (R. B. 
G.) thanks $\ldots$''. Instead, try ``R. B. G. thanks$\ldots$''. Put sponsor 
acknowledgments in the unnumbered footnote on the first page.

\section{Resultados}
The preferred spelling of the word ``acknowledgment'' in America is without 
an ``e'' after the ``g''. Avoid the stilted expression ``one of us (R. B. 
G.) thanks $\ldots$''. Instead, try ``R. B. G. thanks$\ldots$''. Put sponsor 
acknowledgments in the unnumbered footnote on the first page.

\section{Conclusões}
The preferred spelling of the word ``acknowledgment'' in America is without 
an ``e'' after the ``g''. Avoid the stilted expression ``one of us (R. B. 
G.) thanks $\ldots$''. Instead, try ``R. B. G. thanks$\ldots$''. Put sponsor 
acknowledgments in the unnumbered footnote on the first page.

\begin{thebibliography}{00}
\bibitem{patterson2017} D. A. Patterson and J. L. Hennessy, \textit{Organização e Projeto de Computadores: Interface Hardware/Software}, 4th ed. Rio de Janeiro: Elsevier, 2017.
\end{thebibliography}

\end{document}
